% Francais
Le but de ce travail de Bachelor est de mettre en place une plateforme web dédiée à la demande de travaux pour le FabLab.
Actuellement, la procédure est mal définie : les échanges se font manuellement par email et il n'y a pas de suivi des commandes traçable, ce qui peut impliquer des confusions, des retards de fabrication, etc.

Avec cette nouvelle plateforme, la procédure sera claire et automatisée.
Les techniciens et les clients pourront communiquer entre eux en temps réel et auront la possibilité d'avoir une vue de l'avancement du projet.
Un système de notification personnalisable sera disponible pour informer rapidement les utilisateurs concernant l'avancement de leurs demandes.
De plus, le système sera réactif : il sera accessible et ergonomique autant sur un PC que sur un smartphone.

L'interface utilisera l'authentification via Switch AAI, le même service qui est utilisé pour se connecter à son compte GAPS ou Cyberlearn, ce qui signifie que tous les étudiants y auront directement accès, sans avoir à se créer un compte.

\asterism

% English
The goal of this Bachelor's thesis is to create a web platform for submitting job requests to the FabLab.
Currently, the procedure is not well defined : communication is manually done by email and there is no way to monitor the job's progress, wich may cause confusions, production delays, etc.

With this new platform, the procedure will be clear and automated.
Both technicians and clients will be able to communicate in real time and have access to an overview of the job's progress.
A customizable notification system will be available to quickly inform the users about their job's progress.
Furthermore, the system will be reactive : it will be available and adaptive for both PC and smartphone screens.

The interface will use Switch AAI's authentication, the same service that is used to connect to GAPS or Cyberlearn, wich means that all students will have direct access, without the need to create an account.